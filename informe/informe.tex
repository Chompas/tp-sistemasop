%% LyX 2.0.2 created this file.  For more info, see http://www.lyx.org/.
%% Do not edit unless you really know what you are doing.
\documentclass[spanish]{article}
\usepackage[T1]{fontenc}
\usepackage[latin9]{inputenc}
\pagestyle{empty}
\usepackage{babel}
\addto\shorthandsspanish{\spanishdeactivate{~<>}}

\usepackage{float}
\usepackage[unicode=true,
 bookmarks=true,bookmarksnumbered=false,bookmarksopen=false,
 breaklinks=false,pdfborder={0 0 1},backref=false,colorlinks=false]
 {hyperref}
\hypersetup{pdftitle={Sistemas Operativos},
 pdfauthor={Maria Ines Parnisari}}

\makeatletter
%%%%%%%%%%%%%%%%%%%%%%%%%%%%%% User specified LaTeX commands.
\usepackage{fancyhdr}% http://ctan.org/pkg/fancyhdr
\fancyhead{}% Clear all headers
\fancyfoot{}% Clear all footers
\fancyfoot[R]{\thepage}% Place "num page" in Right of footer
\fancyfoot[L]{Grupo 2 - Tema A}
\renewcommand{\headrulewidth}{0pt}% Remove header rule
%\renewcommand{\footrulewidth}{0pt}% Remove footer rule
\pagestyle{fancy}% Set page style to "fancy"

\makeatother

\begin{document}

\title{{[}75.08{]} Sistemas Operativos\\
Trabajo Pr�ctico}

\maketitle
\tableofcontents{}

\pagebreak{}


\section{Hip�tesis y Aclaraciones Globales}
\begin{itemize}
\item \texttt{Instalar\_TP }

\begin{itemize}
\item El log de instalaci�n es �nico. Todas las instalaciones loguear�n
en el mismo archivo dentro de \texttt{\$CONFDIR}.
\item Recibe un �nico par�metro con la ruta del GRUPO. Ejemplo: \texttt{./Instalar\_TP.sh
\textquotedbl{}path/hasta/el/grupo\textquotedbl{}}.
\end{itemize}
\item \texttt{Grabar\_L }

\begin{itemize}
\item Recibe un par�metro opcional \texttt{-i} para invocarlo s�lo desde
el modulo de instalaci�n.
\item Cuando el archivo de log es muy grande se quitan todas las l�neas,
excepto las �ltimas 5.
\item Fecha y hora en formato DD-MM-AA HH:MM:SS (script aparte?)
\end{itemize}
\item \texttt{Mover\_A }

\begin{itemize}
\item El n�mero de secuencia \texttt{nnn} es descentralizado (no es com�n
para todo el sistema). Se lo obtiene directamente del directorio de
duplicados, por eso no es necesario guardarlo.
\end{itemize}
\item \texttt{Imprimir\_A}

\begin{itemize}
\item Se utiliza una estructura de hash para manejar las opciones y las
funciones a las que est�n asociadas dichas opciones. 
\item Si se especifica la opci�n \textquotedbl{}\texttt{t}\textquotedbl{},
el archivo se escribe siempre, independientemente de si se especific�
\textquotedbl{}\texttt{-w}\textquotedbl{} como argumento. El separador
del archivo escrito es \textquotedbl{};\textquotedbl{}. 
\end{itemize}
\end{itemize}

\section{Problemas Relevantes}
\begin{itemize}
\item Problema:
\item Soluci�n:
\item etc
\end{itemize}

\section{Archivo README}

\include{/home/maine/Desktop/tp-sistemasop/README}


\section{Esquema de ejecuci�n}

TO DO: esquema de secuencia de los comandos. Explicar c�mo se vinculan
entre s� y qu� input/output manejan.


\section{Nuevas Funciones y/o Comandos Auxiliares}
\begin{itemize}
\item Nombre de la funci�n:

\begin{itemize}
\item Utilizada por:
\item Uso:
\end{itemize}
\end{itemize}

\section{Nuevos Archivos}
\begin{itemize}
\item Nombre de archivo:

\begin{itemize}
\item Temporal/Permanente
\item Almacenado en:
\item Utilizado por:
\item Uso:
\end{itemize}
\end{itemize}

\section{Listado de Datos}

TO DO: incluir SALAS, OBRAS, COMBOS, RESERVAS.


\section{Hoja de Ruta ``Camino Feliz''}

\pagebreak{}


\section*{Ap�ndice}

(Enunciado)
\end{document}
