%% LyX 2.0.6 created this file.  For more info, see http://www.lyx.org/.
%% Do not edit unless you really know what you are doing.
\documentclass[spanish]{article}
\usepackage[T1]{fontenc}
\usepackage[latin9]{inputenc}
\usepackage{geometry}
\geometry{verbose,tmargin=3cm,bmargin=3cm,lmargin=3cm,rmargin=3cm}
\pagestyle{empty}
\usepackage{babel}
\addto\shorthandsspanish{\spanishdeactivate{~<>}}

\usepackage{array}
\usepackage{float}
\usepackage[unicode=true,
 bookmarks=true,bookmarksnumbered=false,bookmarksopen=false,
 breaklinks=false,pdfborder={0 0 1},backref=false,colorlinks=false]
 {hyperref}
\hypersetup{pdftitle={Sistemas Operativos},
 pdfauthor={Maria Ines Parnisari}}

\makeatletter

%%%%%%%%%%%%%%%%%%%%%%%%%%%%%% LyX specific LaTeX commands.
%% Because html converters don't know tabularnewline
\providecommand{\tabularnewline}{\\}

\@ifundefined{date}{}{\date{}}
%%%%%%%%%%%%%%%%%%%%%%%%%%%%%% User specified LaTeX commands.
\usepackage{fancyhdr}% http://ctan.org/pkg/fancyhdr
\fancyhead{}% Clear all headers
\fancyfoot{}% Clear all footers
\fancyfoot[R]{\thepage}% Place "num page" in Right of footer
\fancyfoot[L]{Grupo 1 - Tema A}
\renewcommand{\headrulewidth}{0pt}% Remove header rule
%\renewcommand{\footrulewidth}{0pt}% Remove footer rule
\pagestyle{fancy}% Set page style to "fancy"

\makeatother

\begin{document}

\title{Trabajo Pr�ctico de Sistemas Operativos}

\maketitle
\textbf{\large{Tema}}{\large{: A}}{\large \par}

\textbf{\large{Grupo}}{\large{: 1}}{\large \par}

\textbf{\large{Ayudante}}{\large{:}}{\large \par}

\begin{center}
{\large{}}%
\begin{tabular}{|c|c|}
\hline 
\multicolumn{2}{|c}{{\large{Evaluaci�n grupal}}}\tabularnewline
\hline 
\hline 
{\large{Nota}} & \tabularnewline
\hline 
{\large{Fecha}} & \tabularnewline
\hline 
{\large{Firma}} & \tabularnewline
\hline 
\end{tabular}
\par\end{center}{\large \par}

\textbf{\large{Integrantes}}{\large{:}}{\large \par}

\begin{center}
{\large{}}%
\begin{tabular}{|c|c|c|>{\centering}p{2cm}|>{\centering}p{2cm}|>{\centering}p{2cm}|}
\hline 
 & {\large{Padr�n}} & {\large{Apellido, nombre}} & {\large{Asistencia a entrega}} & {\large{Asistencia a revisi�n}} & {\large{Evaluaci�n individual final}}\tabularnewline
\hline 
\hline 
{\large{1}} & {\large{87165}} & {\large{Tarcetti, Lucas Damian}} &  &  & \tabularnewline
\hline 
{\large{2}} & {\large{85840}} & {\large{Viscarra, Carlos Emiliano}} &  &  & \tabularnewline
\hline 
{\large{3}} & {\large{89683}} & {\large{Castarataro, Pablo}} &  &  & \tabularnewline
\hline 
{\large{4}} & {\large{86651}} & {\large{Zelechowski, Sergio}} &  &  & \tabularnewline
\hline 
{\large{5}} & {\large{93343}} & {\large{Lara, Javier }} &  &  & \tabularnewline
\hline 
{\large{6}} & {\large{92235}} & {\large{Parnisari, Mar�a In�s }} &  &  & \tabularnewline
\hline 
\end{tabular}
\par\end{center}{\large \par}

\pagebreak{}

\textbf{\large{Tema}}{\large{: A}}{\large \par}

\textbf{\large{Grupo}}{\large{: 1}}{\large \par}

\textbf{\large{Ayudante}}{\large{:}}{\large \par}

\begin{center}
{\large{}}%
\begin{tabular}{|c|c|}
\hline 
\multicolumn{2}{|c}{{\large{Evaluaci�n grupal}}}\tabularnewline
\hline 
\hline 
{\large{Nota}} & \tabularnewline
\hline 
{\large{Fecha}} & \tabularnewline
\hline 
{\large{Firma}} & \tabularnewline
\hline 
\end{tabular}
\par\end{center}{\large \par}

\textbf{\large{Integrantes}}{\large{:}}{\large \par}

\begin{center}
{\large{}}%
\begin{tabular}{|c|c|c|>{\centering}p{2cm}|>{\centering}p{2cm}|>{\centering}p{2cm}|}
\hline 
 & {\large{Padr�n}} & {\large{Apellido, nombre}} & {\large{Asistencia a entrega}} & {\large{Asistencia a revisi�n}} & {\large{Evaluaci�n individual final}}\tabularnewline
\hline 
\hline 
{\large{1}} & {\large{87165}} & {\large{Tarcetti, Lucas Damian}} &  &  & \tabularnewline
\hline 
{\large{2}} & {\large{85840}} & {\large{Viscarra, Carlos Emiliano}} &  &  & \tabularnewline
\hline 
{\large{3}} & {\large{89683}} & {\large{Castarataro, Pablo}} &  &  & \tabularnewline
\hline 
{\large{4}} & {\large{86651}} & {\large{Zelechowski, Sergio}} &  &  & \tabularnewline
\hline 
{\large{5}} & {\large{93343}} & {\large{Lara, Javier }} &  &  & \tabularnewline
\hline 
{\large{6}} & {\large{92235}} & {\large{Parnisari, Mar�a In�s }} &  &  & \tabularnewline
\hline 
\end{tabular}
\par\end{center}{\large \par}

\pagebreak{}

\tableofcontents{}

\pagebreak{}


\section{Hip�tesis y Aclaraciones Globales}
\begin{itemize}
\item \texttt{Instalar\_TP }

\begin{itemize}
\item El log de instalaci�n es �nico. Todas las instalaciones loguear�n
en el mismo archivo dentro de \texttt{\$CONFDIR}.
\item Recibe un �nico par�metro con la ruta del GRUPO. Ejemplo: \texttt{./Instalar\_TP.sh
\textquotedbl{}path/hasta/el/grupo\textquotedbl{}}.
\item El instalador buscar� los archivos 'obras.mae', 'salas.mae' en la
carpeta 'mae'. Adem�s buscar� el archivo 'combos.disp' en la carpeta
'disp'.
\end{itemize}
\item \texttt{Iniciar\_A}

\begin{itemize}
\item Antes de verificar que la configuraci�n est� correcta, se escribe
en el log de instalaci�n. Luego, se escribe en el log com�n.
\end{itemize}
\item \texttt{Grabar\_L }

\begin{itemize}
\item Recibe un par�metro opcional \texttt{-i} para invocarlo s�lo desde
el modulo de instalaci�n.
\item Cuando el archivo de log es muy grande se quitan todas las l�neas,
excepto las �ltimas 5.
\item Fecha y hora en formato DD-MM-AA HH:MM:SS (script aparte?)
\end{itemize}
\item \texttt{Mover\_A }

\begin{itemize}
\item El n�mero de secuencia \texttt{nnn} es descentralizado (no es com�n
para todo el sistema). Se lo obtiene directamente del directorio de
duplicados, por eso no es necesario guardarlo.
\end{itemize}
\item \texttt{Imprimir\_A}

\begin{itemize}
\item Se usa un array asociativo (estructura de hash) para:

\begin{itemize}
\item Manejar las opciones y las funciones a las que est�n asociadas dichas
opciones. 
\item Manejar los eventos candidatos para la opci�n \textquotedbl{}-i\textquotedbl{}.
\end{itemize}
\item Si se especifica la opci�n \textquotedbl{}\texttt{t}\textquotedbl{},
el archivo se escribe siempre, independientemente de si se especific�
\textquotedbl{}\texttt{-w}\textquotedbl{} como argumento. El separador
del archivo escrito es \textquotedbl{};\textquotedbl{}. 
\item Para no permitir m�s de una ejecuci�n en simult�neo, se utiliza un
archivo temporal llamado \texttt{Imprimir\_A.PID} que almacena el
PID del proceso corriendo.
\end{itemize}
\end{itemize}

\section{Problemas Relevantes}
\begin{itemize}
\item Problema: manejo de fechas
\item Soluci�n:
\item etc
\end{itemize}

\section{Archivo README}

\include{/home/maine/Desktop/tp-sistemasop/README}


\section{Esquema de ejecuci�n}

TO DO: esquema de secuencia de los comandos. Explicar c�mo se vinculan
entre s� y qu� input/output manejan.


\section{Nuevas Funciones y/o Comandos Auxiliares}
\begin{itemize}
\item Nombre de la funci�n:

\begin{itemize}
\item Utilizada por:
\item Uso:
\end{itemize}
\end{itemize}

\section{Nuevos Archivos}
\begin{itemize}
\item Nombre de archivo: Imprimir\_A.PID

\begin{itemize}
\item Temporal
\item Almacenado en: /tmp
\item Utilizado por: Imprimir\_A
\item Uso: permitir solo una instancia en ejecuci�n del programa.
\end{itemize}
\end{itemize}

\section{Listado de Datos}

TO DO: incluir SALAS, OBRAS, COMBOS, RESERVAS.


\section{Hoja de Ruta de Prueba ``Camino Feliz''}

\pagebreak{}


\section*{Ap�ndice}

(Enunciado)
\end{document}
